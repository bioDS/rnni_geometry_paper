\documentclass[11pt]{amsart}


\usepackage[ibidtracker=false,uniquename=false,giveninits=true,terseinits=true,backend=biber]{biblatex}
\usepackage{float}
\usepackage{graphicx}
\usepackage{todonotes}
\usepackage{subcaption}
\usepackage{amsmath}
\usepackage{amsthm}
\usepackage{amssymb}
\usepackage{algorithm}
\usepackage[noend]{algorithmic}
\usepackage[foot]{amsaddr}
\usepackage[misc]{ifsym}
\usepackage{enumitem}
\usepackage{geometry}
\usepackage[hidelinks]{hyperref}

\renewbibmacro{in:}{}
\addbibresource{rnni_geometry.bib}
\AtEveryBibitem{
	\clearlist{language}
}

\setlist{leftmargin = 0pt}
\geometry{margin=1in}

\newtheorem{proposition}{Proposition}
\newtheorem{theorem}{Theorem}
\newtheorem{lemma}{Lemma}
\newtheorem{corollary}{Corollary}
\newtheorem{problem}{Problem}
\newtheorem{conjecture}{Conjecture}

\newcommand{\tocite}{ {\color{red}\fbox{CITATION}} }
\newcommand{\rnni}{\mathrm{RNNI}}
\newcommand{\findpath}{\textsc{FindPath}}
\newcommand{\mrca}{\mathrm{mrca}}
\newcommand{\rank}{\mathrm{rank}}
\newcommand{\nni}{\mathrm{NNI}}
\newcommand{\spr}{\mathrm{SPR}}
\newcommand{\tbr}{\mathrm{TBR}}
\newcommand{\fp}{\mathrm{FP}}
\newcommand{\dtt}{\mathrm{DtT}}
\newcommand{\np}{\mathbf{NP}}
\newcommand{\decprob}[1]{\rnni(#1)\text{-}\mathrm{SP}}
\newcommand{\rad}{\mathit{rad}}
\renewcommand{\O}{\mathcal O}
\renewcommand{\epsilon}{\varepsilon}
\renewcommand{\thesubfigure}{\Alph{subfigure}}

\newcommand{\summary}[1]{\textbf{#1}} % Print summaries to .pdf
% \newcommand{\summary}[1]{} % Hide summaries in .pdf

\graphicspath{{figures/}}

\sloppy


\title[Geometry of ranked tree spaces]{The Geometry of the Ranked Nearest Neighbour Interchange Space}
\date{\today}
\author{Lena Collienne}
\email{lcollienne@cs.otago.ac.nz}
\address{Department of Computer Science, University of Otago, New Zealand}
\author{Alex Gavryushkin\textsuperscript{\Letter}}
\email{\textsuperscript{\Letter}alex@biods.org}
\thanks{}

\begin{document}

\begin{abstract}
\end{abstract}

\maketitle


\section{Introduction}

\summary{Motivation -- ranked trees and tree rearrangements.}

\summary{More motivation, sepcifically for $\rnni$, $\rnni(\rho)$, and DtT}

\summary{Why we want to investigate geometrical properties of $\rnni(\rho)$ for different $\rho$ and that we want to compare properties in spaces for different $\rho$.}

\summary{Structure of the paper.}


\section{Technical Introduction}

\summary{Defining the basics of ranked trees.}

Things we need to define:
\begin{itemize}
	\item with $\rnni$ we mean $\rnni(1)$ -- do we want to use this everywhere?
	\item $d(T,R)$ is the weight of a shortest path
	\item caterpillar path and shortest caterpillar path (do we want to define caterpillar trees here or in the caterpillar tree section?)
	\item convexity
	\item representation of caterpillar trees as lists of leaves instead of cluster representation
	\item $\rnni(\infty)$
	\item rank interval for Lemma~\ref{lemma:nni_path_to_caterpillar}
	\item cherry (for lemma~\ref{lemma:max_dist_ctree})
	\item one-neighbourhood (for Theorem~\ref{thm:caterpillar_convex_rnni})
	\item Discrete time trees, length moves, and that this is different from how it is used in previous paper
\end{itemize}

\summary{Introduce $\findpath$ and that it computes $\rnni$ distances.}


\section{$\rnni$}

\subsection{Caterpillar Trees}

\summary{The set of caterpillar trees is convex in $\rnni$.}
\begin{theorem}
	The set of caterpillar trees is convex in $\rnni$.
	\label{thm:caterpillar_convex_rnni}
\end{theorem}

\begin{proof}
	Let $T$ and $R$ be two caterpillar trees.
	For proving the lemma, we show that there is a tree caterpillar tree $T'$ in the one-neighbourhood with distance $d(T',R) = d(T,R) - 1$ to $R$.
	The existence of a path between $T$ and $R$ that only consists of caterpillar trees then follows inductively.
	If there is a leaf that is part of the cherry in both trees $T$ and $R$, then the first tree on $\fp(T,R)$ is a caterpillar tree, which proves the statement.
	So we only need to consider the case that $T$ and $R$ have different cherries.
	Let $a_1$ and $a_2$ be the leaves of the cherry of $T$ and $a_3$ the leaf which has parent of rank two in $T$.
	Since neither $a_1$ nor $a_2$ are in the cherry of $R$, their parents must be different and therefore have different ranks in $R$.
	For the same reason, the parents of $a_1$ and $a_2$ also have a rank different to the parent of $a_3$ in $R$.
	% Two cases: $a_3$ above $a_1$ and $a_2$ in $R$ or one of $\{a_1,a_2\}$ is above $a_3$.
	\todo{Finish this proof!} 
\end{proof}


\summary{With Theorem~\ref{thm:caterpillar_convex_rnni} we can find a more efficient way to compute distances between caterpillar trees.}

\begin{theorem}
	Let $T$ be an arbitrary caterpillar tree and $R$ the caterpillar tree $[1,2, \ldots, n]$.
	\todo{Note somewhere that we refer to the list representations of caterpillar trees by $T$, for example}
	Then it is
	\[d(T,R) = p_T - m_T,\]
	where $p_T$ is the number of transpositions in $T$ (Kendall-tau distance\tocite) and
	\[m_T := |\{i \in \{1, \ldots, n\}| (i < k\ \forall k \preceq i) \text{ and } i \preceq 1 \text{ and } i \preceq 2\}|\]
	\todo{define the partial order $\preceq$}
	\label{thm:caterpillar_distance_formula}
\end{theorem}

\begin{proof}

\end{proof}

\subsection{Diameter and Radius}

%TODO: Make this independent of rho
\summary{Definition of Diameter and that we consider it depending on $\rho$.}
In this section we want to investigate the diameter of $\rnni(\rho)$, which is the greatest distance between any pair of vertices (trees) in this graph, i.e. $\max\limits_{\text{trees }T,R}d(T,R)$.
We will show that the diameter of $\rnni(\rho)$ depends on $\rho$.
Since the $\rnni$ graph, where rank moves have weight one, is the only one for which a polynomial time algorithm for computing distances is known, we start considering the diameter of this graph before turning to $\rnni(\rho)$ for different values of $\rho$.

\summary{Diameter in $\rnni$ follows from results in previous paper.}
The algorithm $\findpath$ of \autocite{collienne2020computing}, which computes distances in $\rnni$, facilitates finding the diameter of the $\rnni$ graph. 
\begin{corollary}
	The diameter of $\rnni$ is $\frac{(n-1)(n-2)}{2}$.
	\label{cor:diameter_rnni}
\end{corollary}

\begin{proof}
	The proof of Corollary 1 in \autocite{collienne2020computing} gives an example of trees for which the length of the path computed by $\findpath$, and therefore the distance, is $\frac{(n-1)(n-2)}{2}$.
	It remains to show that $\findpath$ cannot computes paths longer than $\frac{(n-1)(n-2)}{2}$.
	\todo{Explain that this cannot happen by explicitely mention how often while and for loop are executed -- do we want to repeat the pseudo-code in this paper?}
\end{proof}

%TODO: This is mainly relevant for RNNI(\infty). As it is a statement about RNNI, we could still keep it here, if needed
\begin{lemma}
	In $\rnni$ every tree is connected to a caterpillar tree by a path that consists of $\nni$ moves only.
	\label{lemma:nni_path_to_caterpillar}
\end{lemma}

\begin{proof}
	We prove this lemma for a given tree $T$ by induction on the number $i$ of rank intervals of $T$.
	In the base case $k = 0$ the tree $T$ is a caterpillar tree and the statement is true.
	For the induction step we assume that $T$ has $i+1$ rank intervals and all trees with less than $i+1$ rank intervals are connected to a caterpillar trees through a sequence of $\nni$ moves.
	We now construct a series of $\nni$ moves that, starting at $T$, ends in a tree that has less than $i+1$ rank intervals.
	Let the node of rank $k+1$ in $T$ be the highest ranked node incident to a rank interval.
	This means that the interval given by nodes of rank $k$ and $k+1$ is a rank interval and all intervals above $(T)_{k+1}$ are edges.
	Let $(T)_l$ be the parent of $(T)_k$.
	Hence $l > k+1$, and $[(T)_{l-1}, (T)_{l}]$ is an edge.
	On this edge an $\nni$ move can be performed that results in a tree $T'$ in which the rank of the parent of $(T')_k$ is $l-1$, as depicted in Figure~\ref{fig:nni_path_caterpillar}.
	In the tree $R$ resulting from iteratively repeating this until $l = k+1$ the parent $(R)_k$ has rank $k+1$.
	Hence the rank interval bounded by nodes of rank $k$ and $k+1$ in $T$ turned into an edge while all intervals above $k+1$ remain edges in $R$.
	As $R$ has one rank interval less than the $i+1$ rank intervals of $T$, the induction hypothesis can be applied to $R$.
	This gives a sequence from $T$ to a caterpillar via $R$ that consists of $\nni$ moves only.
	\begin{figure}[ht]
		\includegraphics[width=0.8\textwidth]{nni_path_caterpillar.eps}
		\caption{Example of a tree $T$ with two rank intervals and a sequence of $\nni$ moves that results in a tree $R$ with one rank interval as described in the proof of Lemma~\ref{lemma:nni_path_to_caterpillar}.}
		\label{fig:nni_path_caterpillar}
	\end{figure}
\end{proof}

With this lemma we are able to prove that the diameter of $\rnni(\infty)$ is quadratic in $n$.
\begin{corollary}
	The diameter of $\rnni(\infty)$ is less or equal to $3 \frac{(n-1)(n-2)}{2}$.
\end{corollary}

\begin{proof}
	Let $T$ and $R$ be two trees in $\rnni(\infty)$.
	With Lemma~\ref{lemma:nni_path_to_caterpillar} we know that any tree can be connected to a caterpillar tree by a sequence of $\nni$ moves.
	Moreover, the proof of the lemma implicitly proposes an algorithm to convert a tree into a caterpillar tree by removing rank intervals by a top-down approach.
	We show that the paths from a tree to a caterpillar tree that are given by this algorithm have at most length $\frac{(n-1)(n-1)}{2}$.
	Since we also know that the set of caterpillar trees is convex in $\rnni$ (Theorem~\ref{thm:caterpillar_convex_rnni}), and hence the maximum distance between caterpillar trees is bounded by the diameter $\frac{(n-1)(n-2)}{2}$ of $\rnni$ (Corollary~\ref{cor:diameter_rnni}), this corollary follows.

	For finding the maximum length of a path computed by the algorithm suggested in the proof of Lemma~\ref{lemma:nni_path_to_caterpillar}, we consider every iteration (induction step).
	In each of these, at least one rank interval turns into an edge.
	The number of $\nni$ moves needed for this depends on the rank of the upper node bounding the rank interval.
	If we assume the worst case, that is, every interval except for the edge $[(T)_{n-1},(T)_{n-2}]$ is a rank interval, then there are $i$ $\nni$ moves needed in iteration $i$.
	Since there are $n-2$ iterations, it follows that at most $\frac{(n-1)(n-2)}{2}$ $\nni$ moves are needed to get from an arbitrary tree to a caterpillar tree.
	\todo{I suspect writing the algorithm down properly and using it to prove Lemma~\ref{lemma:nni_path_to_caterpillar} and this corollary might be easier than this. This explanation of the 'implicitly' defined algorithm is not sufficient.}
\end{proof}

\summary{Radius of $\rnni$ is equal to its diameter.}
For proving that the radius of $\rnni$, which is defined $\min\limits_{\text{tree } T}\max\limits_{\text{tree }R} d(T,R)$, equals its diameter, we need the following lemma.
\todo{Can we come up with a more intuitive description of what the radius of a graph is?}

\begin{lemma}
	For every tree on $n$ leaves exists a caterpillar tree with distance $\frac{(n-1)(n-2)}{2}$ from it in $\rnni$.
	\label{lemma:max_dist_ctree}
\end{lemma}

\begin{proof}
	We prove this lemma by induction on the number of leaves $n$.
	The base case $n=3$ is trivial, as all three trees in this space are caterpillar trees with distance one.
	For the induction step we consider an arbitrary tree $T$ with $n + 1$ leaves.
	Let $T'$ be the tree on $n$ leaves resulting from deleting one of the cherry leaves, which we denote by $x$, of $T$.
	By applying the induction hypothesis on $T'$ we can find a caterpillar tree $R'$ that has distance $\frac{(n-1)(n-2)}{2}$ to $T'$.
	Now consider the tree $R$ resulting from adding the leaf $x$ at the top of $R'$ such that the root of $R$ has $x$ and $R'$ as children.

	In the first iteration of $\findpath$ the leaf $x$ moves down by $\nni$ moves until it reaches the second cherry leaf $y$ of $T$ and builds a cherry with it, which then is moved down by rank swaps as depicted in Figure~\ref{fig:max_dist_ctree}.
	There are $n-1$ $\rnni$ moves needed for this as $(x,y)_R$ moves down from the root with rank $n$ to the cherry of rank one.
	The tree at the end of this first iteration on $\fp(T,R)$ is equals $R'$ when ignoring the leaf $x$ and its parent.
	Since the cluster $\{x,y\}$ is not considered again in $\findpath$, the remaining part of $\fp(T,R)$ contains the same moves as $\fp(T',R')$ from which we can follow that $|\fp(T,R)| = |\fp(T',R')| + n-1$.
	\todo{Is it OK to phrase it like this?}
	Therefore it is $d(T,R) = \frac{(n-1)(n-2)}{2} + n-1 = \frac{n(n-1)}{2}$, which proves the lemma.
	\begin{figure}[ht]
		\includegraphics[width=0.8\textwidth]{max_dist_ctree.eps}
		\caption{Initial $n - 1$ $\rnni$ moves of $\fp(R,T)$ as described in the proof of Lemma~\ref{lemma:max_dist_ctree}.
		Removing the leaf $x$ and suppressing the non-root node of degree two from the tree on the right results in $R'$ as described in the lemma.}
		\label{fig:max_dist_ctree}
	\end{figure}
\end{proof}

\begin{proposition}
	The radius of the $\rnni$ graph equals its diameter, i.e. $\rad(\rnni) = \frac{(n-1)(n-2)}{2}$.
\end{proposition}

\begin{proof}
	From Lemma~\ref{lemma:max_dist_ctree} we know that there is a tree with distance $\frac{(n-1)(n-2)}{2}$ from any given tree.
	Hence the radius equals the diameter of $\rnni$.
\end{proof}


\subsection{Cluster Property}

\summary{Definition of Cluster Property and why it is relevant.}

\summary{$\rnni$ has the cluster property.}
\begin{theorem}
	The $\rnni$ graph has the cluster property.
\end{theorem}


\section{$\rnni(\rho)$}

\subsection{Caterpillar Trees}

\summary{The set of caterpillar trees is convex in $\rnni(\rho)$ for $\rho > 1$.}
\begin{corollary}
	The set of caterpillar trees is convex in $\rnni(\rho)$ if and ony if $\rho \geq 1$.
\end{corollary}

\begin{proof}
	We start with proving that if the set of caterpillar trees is convex, then it is $\rho \geq 1$.
	More specifically, we show that if $\rho < 1$, then the set of caterpillar trees is not convex, from which the previous statement follows.
	Therefore we consider the following example of trees with four leaves.
	Let
	\begin{align*}
		T &= [\{1,2\},\{1,2,3\},\{1,2,3,4\}]\text{ and }\\
		R &= [\{3,4\},\{1,3,4\},\{1,2,3,4\}].
	\end{align*}
	The shortest among all caterpillar paths between $T$ and $R$ has length three and is depicted at the bottom of Figure~\ref{fig:caterpillar_non_convex}.
	However, there is a path that is shorter than this path and has length $2 + \rho$, which is smaller than three for $\rho < 1$.
	This path is depicted at the top of Figure~\ref{fig:caterpillar_non_convex}.
	Hence, the set of caterpillar trees is not convex if $\rho < 1$.
	\begin{figure}[ht]
		\includegraphics[width=0.5\textwidth]{caterpillar_non_convex.eps}
		\caption{A shortest caterpillar path between caterpillar trees $T$ and $R$ at the bottom and a path consisting of non-caterpillar trees at the top.
		The path at the top is shorter than the one at the bottom for all $\rho<1$.}
		\label{fig:caterpillar_non_convex}
	\end{figure}

	It remains to prove that if $\rho \geq 1$, then the set of caterpillar trees is convex.
	With Theorem~\ref{thm:caterpillar_convex_rnni} we know that this is true for $\rho = 1$.
	Moreover, the statement for $\rho > 1$ follows from the same theorem as described in the following, where we assume that $\rho > 1$.
	All paths between any two trees in $\rnni(\rho)$ are longer or have equal length to the path containing the same moves in $\rnni$, as the weights of the moves in $\rnni(\rho)$ are greater or equal to the weights of the same moves in $\rnni$.
	Specifically, the only paths that have the same length in $\rnni$ and $\rnni(\rho)$ are paths that contain only $\nni$ moves.
	It follows that caterpillar paths of $\rnni$ have the same length in $\rnni(\rho)$, while other paths might be longer in $\rnni(\rho)$, but cannot be shorter than the same path in $\rnni$.s
	And as there is a shortest caterpillar path between all caterpillar trees in $\rnni$, this path is shortest path in $\rnni(\rho)$ as well.
	This proves that the set of caterpillar trees is convex in $\rnni(\rho)$ for $\rho \geq 1$, which completes the proof.
\end{proof}

\subsection{Diameter and Radius}

\summary{Diameter for $\rnni(0)$ follows from $\nni$}
Not only for $\rnni$, but also for $\rnni(0)$, we know the diameter from previous results.
As rank moves in $\rnni(0)$ weigh zero, the distance between two trees in this space is the same as the $\nni$ distance between these trees when ignoring ranks.
Therefore, these graphs have the same diameter.

\begin{proposition}
	The diameter of $\rnni(0)$ is $\Theta(n \log(n))$.
	\label{prop:diameter_nni}
\end{proposition}

\begin{proof}
	This follows from the diameter of the $\nni$ graph, which is known \autocite{Semple2003-nj} to be $\Theta(n \log(n))$.
\end{proof}

\summary{Results for $\rnni$ and $\rnni(0)$ give us bounds for the diameters of all spaces with $0 < \rho < 1$.}
With the previous results in Corollary~\ref{cor:diameter_rnni} and Proposition~\ref{prop:diameter_nni} we can infer bounds for diameters of spaces $\rnni(\rho)$ with $0 < \rho < 1$.
A path in $\rnni(\rho)$ between two trees corresponds to paths in $\rnni(0)$ and $\rnni(1)$ that contain the same moves, but have different total length, due to the different weighing of rank moves.
Therefore, the length of such a path is bounded from below by the length of the corresponding path in $\rnni(0)$ and from above by the corresponding path in $\rnni(1)$.
With Corollary~\ref{cor:diameter_rnni} and Proposition~\ref{prop:diameter_nni} it follows that the diameter of $\rnni(\rho)$ with $\rho < 1$ is bounded from below by $\Theta(n \log(n))$ and from above by $\frac{(n-1)(n-2)}{2}$.

\summary{Diameter of $\rnni(\rho)$ for $\rho > 1$.}
So far $\rnni(\rho)$ for $0 \leq \rho \leq 1$ has been the centre of our investigation of diameters.
We now continue by considering spaces $\rnni(\rho)$ for $\rho > 1$, where rank moves are more expensive than $\nni$ moves.
Specifically, we give an upper bound for the diameter of $\rnni(\infty)$ from which we can follow that all spaces $\rnni(\rho)$ have a diameter less or equal to this bound.
Before this, however, we need to observe that for $\rnni(\infty)$ every pair of trees is connected by a path consisting of $\nni$ moves only.

\summary{Radius of $\rnni(\rho)$ for other values of $\rho$.}

\subsection{Cluster Property}

\summary{$\rnni(0)$ does not have cluster property.}
\begin{proposition}
	$\rnni(0)$ does not have the cluster property.
\end{proposition}

\summary{Cluster Property of $\rnni(\rho)$ for $\rho \neq 0, 1$?}


%TODO: Leave this as a section or do we want a section (not just subsection) for DtT?
\section{Generalisation}

\summary{All (?) results from $\rnni$ transfer to discrete time-trees.}

\summary{How partition lattices correspond to $\rnni$.}

\summary{Implementation of FP}

\end{document}