\documentclass[11pt]{amsart}


\usepackage[ibidtracker=false,uniquename=false,giveninits=true,terseinits=true,backend=biber]{biblatex}
\usepackage{float}
\usepackage{graphicx}
\usepackage{todonotes}
\usepackage{subcaption}
\usepackage{amsmath}
\usepackage{amsthm}
\usepackage{amssymb}
\usepackage{algorithm}
\usepackage[noend]{algorithmic}
\usepackage[foot]{amsaddr}
\usepackage[misc]{ifsym}
\usepackage{enumitem}
\usepackage{geometry}
\usepackage[hidelinks]{hyperref}

\renewbibmacro{in:}{}
\addbibresource{rnni_geometry.bib}
\AtEveryBibitem{
  \clearlist{language}
}

\setlist{leftmargin = 0pt}
\geometry{margin=1in}

\newtheorem{proposition}{Proposition}
\newtheorem{theorem}{Theorem}
\newtheorem{lemma}{Lemma}
\newtheorem{corollary}{Corollary}
\newtheorem{problem}{Problem}
\newtheorem{conjecture}{Conjecture}

\newcommand{\rnni}{\mathrm{RNNI}}
\newcommand{\findpath}{\textsc{FindPath}}
\newcommand{\mrca}{\mathrm{mrca}}
\newcommand{\rank}{\mathrm{rank}}
\newcommand{\nni}{\mathrm{NNI}}
\newcommand{\spr}{\mathrm{SPR}}
\newcommand{\tbr}{\mathrm{TBR}}
\newcommand{\fp}{\mathrm{FP}}
\newcommand{\np}{\mathbf{NP}}
\newcommand{\decprob}[1]{\rnni(#1)\text{-}\mathrm{SP}}
\renewcommand{\O}{\mathcal O}
\renewcommand{\epsilon}{\varepsilon}
\renewcommand{\thesubfigure}{\Alph{subfigure}}

\newcommand{\summary}[1]{\textbf{#1}} % Print summaries to .pdf
% \newcommand{\summary}[1]{} % Hide summaries in .pdf

\graphicspath{{figures/}}

\sloppy


\title[Geometry of ranked tree spaces]{The Geometry of the Ranked Nearest Neighbour Interchange Space}
\date{\today}
\author{Lena Collienne}
\email{lcollienne@cs.otago.ac.nz}
\address{Department of Computer Science, University of Otago, New Zealand}
\author{Alex Gavryushkin\textsuperscript{\Letter}}
\email{\textsuperscript{\Letter}alex@biods.org}
\thanks{}

\begin{document}

\maketitle

\begin{abstract}
\end{abstract}

\section{Introduction}

\summary{Motivation -- ranked trees and tree rearrangements.}

\summary{More motivation, sepcifically for $\rnni$ and $\rho$.}

\summary{Why we want to investigate geometrical properties of $\rnni(\rho)$ for different $\rho$ and that we want to compare properties in spaces for different $\rho$.}

\summary{Structure of the paper.}

\section{Technical Introduction}

\summary{Defining the basics of ranked trees.}

\summary{Briefly introduce $\findpath$ and that it computes $\rnni$ distances.}

\section{Diameter and Radius}

\summary{Diameter in $\rnni$ follows from results in previous paper.}

\summary{Diameter of $\rnni(\rho)$ for other values of $\rho$.}

\summary{Radius of $\rnni$ is equal to it's diameter.}

\summary{Radius of $\rnni(\rho)$ for other values of $\rho$.}

\section{Cluster Property}

\summary{Why the Cluster Property is relevant.}

\summary{$\rnni$ has the cluster property.}

\summary{$\rnni(\rho)$ for $\rho < 1$ do not have cluster property.}

\summary{$\rnni(\rho)$ for $\rho > 1$ do have cluster property.}

\section{Caterpillar Trees}

\summary{The set of caterpillar trees is convex in $\rnni$.}

\summary{The set of caterpillar trees is not convex in $\rnni(\rho)$ for $\rho = 0$}

\summary{Caterpillar trees for $\rho \neq 0,1$?}


\section{Generalisation to discrete time-trees}

\summary{All (?) results from $\rnni$ transfer to discrete time-trees.}

\end{document}