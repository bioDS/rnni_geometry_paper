\documentclass[11pt]{amsart}


\usepackage[ibidtracker=false,uniquename=false,giveninits=true,terseinits=true,backend=biber]{biblatex}
\usepackage{float}
\usepackage{graphicx}
\usepackage{todonotes}
\usepackage{subcaption}
\usepackage{amsmath}
\usepackage{amsthm}
\usepackage{amssymb}
\usepackage{algorithm}
\usepackage[noend]{algorithmic}
\usepackage[foot]{amsaddr}
\usepackage[misc]{ifsym}
\usepackage{enumitem}
\usepackage{geometry}
\usepackage[hidelinks]{hyperref}

\renewbibmacro{in:}{}
\addbibresource{rnni_geometry.bib}
\AtEveryBibitem{
	\clearlist{language}
}

\setlist{leftmargin = 0pt}
\geometry{margin=1in}

\newtheorem{proposition}{Proposition}
\newtheorem{theorem}{Theorem}
\newtheorem{lemma}{Lemma}
\newtheorem{corollary}{Corollary}
\newtheorem{problem}{Problem}
\newtheorem{conjecture}{Conjecture}

\newcommand{\rnni}{\mathrm{RNNI}}
\newcommand{\findpath}{\textsc{FindPath}}
\newcommand{\mrca}{\mathrm{mrca}}
\newcommand{\rank}{\mathrm{rank}}
\newcommand{\nni}{\mathrm{NNI}}
\newcommand{\spr}{\mathrm{SPR}}
\newcommand{\tbr}{\mathrm{TBR}}
\newcommand{\fp}{\mathrm{FP}}
\newcommand{\np}{\mathbf{NP}}
\newcommand{\decprob}[1]{\rnni(#1)\text{-}\mathrm{SP}}
\newcommand{\rad}{\mathit{rad}}
\renewcommand{\O}{\mathcal O}
\renewcommand{\epsilon}{\varepsilon}
\renewcommand{\thesubfigure}{\Alph{subfigure}}

\newcommand{\summary}[1]{\textbf{#1}} % Print summaries to .pdf
% \newcommand{\summary}[1]{} % Hide summaries in .pdf

\graphicspath{{figures/}}

\sloppy


\title[Geometry of ranked tree spaces]{The Geometry of the Ranked Nearest Neighbour Interchange Space}
\date{\today}
\author{Lena Collienne}
\email{lcollienne@cs.otago.ac.nz}
\address{Department of Computer Science, University of Otago, New Zealand}
\author{Alex Gavryushkin\textsuperscript{\Letter}}
\email{\textsuperscript{\Letter}alex@biods.org}
\thanks{}

\begin{document}

\begin{abstract}
\end{abstract}

\maketitle


\section{Introduction}

\summary{Motivation -- ranked trees and tree rearrangements.}

\summary{More motivation, sepcifically for $\rnni$ and $\rho$.}

\summary{Why we want to investigate geometrical properties of $\rnni(\rho)$ for different $\rho$ and that we want to compare properties in spaces for different $\rho$.}

\summary{Structure of the paper.}


\section{Technical Introduction}

\summary{Defining the basics of ranked trees.}

Things we need to define:
\begin{itemize}
	\item with $\rnni$ we mean $\rnni(1)$.
	\item $d(T,R)$ is the weight of a shortest path
\end{itemize}

\summary{Introduce $\findpath$ and that it computes $\rnni$ distances.}


\section{Diameter and Radius}

\summary{Definition of Diameter and that we consider it depending on $\rho$.}
The first geometrical property of the graphs $\rnni(\rho)$ that we want to investigate is the diameter, which is the greatest distance between any pair of vertices in a graph.
For our graph $\rnni(\rho)$ the diameter is formally defined as $\max\limits_{\text{trees }T,R}d(T,R)$.
We will show that the diameter of $\rnni(\rho)$ depends on $\rho$.
Since the $\rnni$ graph, where rank moves have weight one, is the only one for which a polynomial time algorithm for computing distances is known, we start considering the diameter of this graph before turning to $\rnni(\rho)$ for different values of $\rho$.

\summary{Diameter in $\rnni$ follows from results in previous paper.}
The algorithm $\findpath$ of \autocite{collienne2020computing}, which computes distances in $\rnni$, facilitates finding the diameter of the $\rnni$ graph. 
\begin{corollary}
	The diameter of the $\rnni$ graph is $\frac{(n-1)(n-2)}{2}$.
\end{corollary}

\begin{proof}
	The the proof of Corollary 1 in \autocite{collienne2020computing} gives an example of trees for which the length of the path computed by $\findpath$ is $\frac{(n-1)(n-2)}{2}$.
	For proving this corollary it remains to show that $\findpath$ cannot compute paths longer than $\frac{(n-1)(n-2)}{2}$.
	\todo{Explain that this cannot happen by explicitely mention how often while and for loop are executed -- do we want to repeat the pseudo-code in this paper?}
\end{proof}

As rank moves in $\rnni(0)$ weigh zero, the distance between two trees in this space is the same as the $\nni$ distance between these trees when ignoring ranks.
Therefore, these graphs have the same diameter.

\summary{Diameter of $\rnni(\rho)$ for other values of $\rho$.}
\begin{proposition}
	The diameter of $\rnni(0)$ is $\Theta(n \log(n))$.
\end{proposition}

\begin{proof}
 This follows from the diameter of the $\nni$ graph, which is known \autocite{Semple2003-nj} to be $\Theta(n \log(n))$.
\end{proof}

\summary{Radius of $\rnni$ is equal to its diameter.}
\begin{proposition}
	The radius of the $\rnni$ graph equals its diameter, i.e. $\rad(\rnni) = \frac{(n-1)(n-2)}{2}$.
\end{proposition}

\summary{Radius of $\rnni(\rho)$ for other values of $\rho$.}


\section{Cluster Property}

\summary{Why the Cluster Property is relevant.}

\summary{$\rnni$ has the cluster property.}
\begin{theorem}
	The $\rnni$ graph has the cluster property.
\end{theorem}

\summary{$\rnni(0)$ does not have cluster property.}
\begin{proposition}
	$\rnni(0)$ does not have the cluster property.
\end{proposition}

\summary{Cluster Property of $\rnni(\rho)$ for $\rho \neq 0, 1$?}


\section{Caterpillar Trees}

\summary{The set of caterpillar trees is convex in $\rnni$.}
\begin{theorem}
	The set of caterpillar trees is convex in $\rnni$.
\end{theorem}

\summary{The set of caterpillar trees is convex in $\rnni(\rho)$ for $\rho > 1$.}
\begin{corollary}
	The set of caterpillar trees is convex in $\rnni(\rho)$ if and ony if $\rho \geq 1$.
\end{corollary}


\section{Generalisation}

\summary{All (?) results from $\rnni$ transfer to discrete time-trees.}

\summary{How partition lattices correspond to $\rnni$.}

\end{document}